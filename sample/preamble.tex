% 図形を読み込めるようにする
\usepackage[dvipdfmx]{graphicx}

% 回り込み図形を使う
\usepackage{wrapfig}

% 図の配置を指定する
\usepackage{here}

% リスティングの設定
% \begin{lstlisting}[caption=foo, label=bar] のように使う。
\usepackage{listings}
\lstset{
  basicstyle={\ttfamily},
  identifierstyle={\small},
  commentstyle={\smallitshape},
  keywordstyle={\small\bfseries},
  ndkeywordstyle={\small},
  stringstyle={\small\ttfamily},
  frame={tlRB},
  breaklines=true,
  columns=fixed,
  basewidth=0.5em,
  numberstyle={\scriptsize},
  stepnumber=1,
  numbersep=1zw,
  keepspaces=true,
  lineskip=-0.5ex
}

% 索引を作る
\usepackage{makeidx}
\makeindex
\usepackage[totoc]{idxlayout}

% TODOを入れる
\usepackage{todonotes}


% PDF内部にハイパーリンクをつける
\usepackage[dvipdfmx]{hyperref}
\usepackage{pxjahyper}
\hypersetup{% hyperrefオプションリスト
  setpagesize=false,
  bookmarksnumbered=true,%
  bookmarksopen=true,%
  colorlinks=true,%
  allcolors=blue
}

% 言葉を強調し、かつ索引に登録する
% \term{言葉}
% \term[よみがな]{言葉}
% のいずれかを使う。よみがなを指定した場合、検索にはよみがな順で
% 掲載される。よみがなは{}ではなく[]でくるむことに注意。

\newcommand*{\term}[2][]{%
  {\sffamily\bfseries #2}%
  \ifx\relax#1\relax\index{#2}\else\index{#1@#2}\fi }

% テーブルフォーマット
\usepackage{booktabs}
\usepackage[normalem]{ulem}
\useunder{\uline}{\ul}{}

% キャプションフォーマットの退避
\makeatletter
\let\MYcaption\@makecaption
\makeatother

% サブキャプションを読み込む
\usepackage{subcaption}
\captionsetup{compatibility=false}      % 必要に応じて

% キャプションフォーマットの復帰
\makeatletter
\let\@makecaption\MYcaption
\makeatother