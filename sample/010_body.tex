\chapter{図形埋め込みのサンプル}
\label{sec:figure_sample}

この章では各種形式の図形を埋め込む例を示します。

\section{PDF図形の埋め込み}
\begin{wrapfigure}[9]{O}[0pt]{0\textwidth}
    \includegraphics[width=5cm,pagebox=cropbox,clip]{image/IMGP3933.pdf}
    \caption{PDF図形の埋め込み}\label{embeded_pdf}
\end{wrapfigure}

プロジェクト内部のimage\_srcディレクトリに置かれたPDFファイルは\LaTeX\
処理直前にimageディレクトリにそのままコピーされます。この際、
処理の対象になるのは拡張子が"pdf"であるようなファイルのみです。
Unixファイルシステムは大文字と小文字を区別することに注意してください。
拡張子"PDF"は無視されます。

したがって、\LaTeX\ 文書内部ではimage/filename.pdfとして参照することで
文書内にPDF図形を表示できます。

\section{JPEG図形の埋め込み}

\begin{wrapfigure}[9]{O}[0pt]{0\textwidth}
    \includegraphics[width=5cm,pagebox=cropbox,clip]{image/IMGP3954.pdf}
    \caption{JPEG図形の埋め込み}\label{embeded_jpeg}
\end{wrapfigure}

同様にプロジェクト内部のimage\_srcディレクトリに置かれたJPEGファイルは\LaTeX\
処理直前にPDFファイルに変換されてimageディレクトリにコピーされます。この際、
処理の対象になるのは拡張子が"jpg"であるようなファイルのみです。
Unixファイルシステムは大文字と小文字を区別することに注意してください。
拡張子"JPG"は無視されます。

したがって、\LaTeX\ 文書内部ではimage/filename.pdfとして参照することで
文書内にPDF図形を表示できます。

埋め立て文字。埋め立て文字。埋め立て文字。埋め立て文字。
埋め立て文字。埋め立て文字。埋め立て文字。埋め立て文字。
埋め立て文字。埋め立て文字。埋め立て文字。埋め立て文字。
埋め立て文字。埋め立て文字。埋め立て文字。埋め立て文字。
埋め立て文字。埋め立て文字。埋め立て文字。埋め立て文字。
埋め立て文字。埋め立て文字。埋め立て文字。埋め立て文字。
埋め立て文字。埋め立て文字。埋め立て文字。埋め立て文字。

\section{PNG図形の埋め込み}


\begin{wrapfigure}[10]{O}[0pt]{0\textwidth}
    \includegraphics[width=5cm,pagebox=cropbox,clip]{image/paint1.pdf}
    \caption{PNG図形の埋め込み}\label{embeded_png}
\end{wrapfigure}

また、image\_srcディレクトリに置かれたPNGファイルは\LaTeX\
処理直前にPDFファイルに変換されてimageディレクトリにコピーされます。この際、
処理の対象になるのは拡張子が"png"であるようなファイルのみです。
Unixファイルシステムは大文字と小文字を区別することに注意してください。
拡張子"PNG"は無視されます。

したがって、\LaTeX\ 文書内部ではimage/filename.pdfとして参照することで
文書内にPDF図形を表示できます。


\section{スクリーンショットの埋め込み}

スクリーンショットはPNGで埋め込みます。このセクションの例題は、Windows上でのスクリーンショットを示しています。

スクリーンショットは、\LaTeX 文書を作るうえで厄介な存在です。普通にスクリーンショットを撮って\LaTeX 文書に埋め込んでもきれいな文書にはなりません。

スクリーンショットを文書中できれいに表示する一番簡単な方法は、

「スクリーンショットを撮る際にOSの表示倍率を上げておく」

事です。例えば、Windowsの場合、コントロールパネルの「システム\textgreater ディスプレイ\textgreater 拡大縮小」を使ってあらかじめ200\% 表示にしておけば、スクリーンショットを撮るときに大きな画像になります。これを\LaTeX 文書で使えば、技術文書としては十分な美しさになります。それでも不満なら400\% で表示しておけばいいでしょう。

それぞれの図はスクリーンショットをとったときのWindowsの表示倍率が違います。

\begin{wrapfigure}{r}[0pt]{0.5\textwidth}
    \begin{center}
        \fbox{\includegraphics*[width=50mm]{image/screenshot_100.pdf}}
        \caption{100\%のスクリーンショット} \label{fig:sc100}
    \end{center}
\end{wrapfigure}

埋め立て文字。埋め立て文字。埋め立て文字。埋め立て文字。
埋め立て文字。埋め立て文字。埋め立て文字。埋め立て文字。
埋め立て文字。埋め立て文字。埋め立て文字。埋め立て文字。

埋め立て文字。埋め立て文字。埋め立て文字。埋め立て文字。
埋め立て文字。埋め立て文字。埋め立て文字。埋め立て文字。
埋め立て文字。埋め立て文字。埋め立て文字。埋め立て文字。
埋め立て文字。埋め立て文字。埋め立て文字。埋め立て文字。
埋め立て文字。埋め立て文字。埋め立て文字。埋め立て文字。
埋め立て文字。埋め立て文字。埋め立て文字。埋め立て文字。
埋め立て文字。埋め立て文字。埋め立て文字。埋め立て文字。

\begin{wrapfigure}{r}[0pt]{0.5\textwidth}
    \begin{center}
        \fbox
        {\includegraphics*[width=50mm]{image/screenshot_200.pdf}}
        \caption{200\%のスクリーンショット} \label{fig:sc200}
    \end{center}
\end{wrapfigure}


埋め立て文字。埋め立て文字。埋め立て文字。埋め立て文字。
埋め立て文字。埋め立て文字。埋め立て文字。埋め立て文字。
埋め立て文字。埋め立て文字。埋め立て文字。埋め立て文字。

埋め立て文字。埋め立て文字。埋め立て文字。埋め立て文字。
埋め立て文字。埋め立て文字。埋め立て文字。埋め立て文字。
埋め立て文字。埋め立て文字。埋め立て文字。埋め立て文字。
埋め立て文字。埋め立て文字。埋め立て文字。埋め立て文字。
埋め立て文字。埋め立て文字。埋め立て文字。埋め立て文字。
埋め立て文字。埋め立て文字。埋め立て文字。埋め立て文字。
埋め立て文字。埋め立て文字。埋め立て文字。埋め立て文字。

\begin{wrapfigure}{r}[0pt]{0.5\textwidth}
    \begin{center}
        \fbox
        {\includegraphics*[width=50mm]{image/screenshot_400.pdf}}
        \caption{400\%のスクリーンショット} \label{fig:sc400}
    \end{center}
\end{wrapfigure}



埋め立て文字。埋め立て文字。埋め立て文字。埋め立て文字。
埋め立て文字。埋め立て文字。埋め立て文字。埋め立て文字。
埋め立て文字。埋め立て文字。埋め立て文字。埋め立て文字。





\section{GIF図形の埋め込み}

\begin{wrapfigure}[10]{O}[0pt]{0\textwidth}
    \includegraphics[width=5cm,pagebox=cropbox,clip]{image/paint2.pdf}
    \caption{GIF図形の埋め込み}\label{embeded_gif}
\end{wrapfigure}


さらに、image\_srcディレクトリに置かれたGIFファイルは\LaTeX\
処理直前にPDFファイルに変換されてimageディレクトリにコピーされます。この際、
処理の対象になるのは拡張子が"gif"であるようなファイルのみです。
Unixファイルシステムは大文字と小文字を区別することに注意してください。
拡張子"GIF"は無視されます。

埋め立て文字。埋め立て文字。埋め立て文字。埋め立て文字。
埋め立て文字。埋め立て文字。埋め立て文字。埋め立て文字。
埋め立て文字。埋め立て文字。埋め立て文字。埋め立て文字。

埋め立て文字。埋め立て文字。埋め立て文字。埋め立て文字。
埋め立て文字。埋め立て文字。埋め立て文字。埋め立て文字。
埋め立て文字。埋め立て文字。埋め立て文字。埋め立て文字。
埋め立て文字。埋め立て文字。埋め立て文字。埋め立て文字。
埋め立て文字。埋め立て文字。埋め立て文字。埋め立て文字。
埋め立て文字。埋め立て文字。埋め立て文字。埋め立て文字。
埋め立て文字。埋め立て文字。埋め立て文字。埋め立て文字。


\section{draw.io図形の埋め込み}

\begin{wrapfigure}[10]{O}[0pt]{0\textwidth}
    \includegraphics[width=3cm,pagebox=cropbox,clip]{image/diagram.pdf}
    \caption{draw.io図形の埋め込み}\label{embeded_drawio}
\end{wrapfigure}

さらに、image\_srcディレクトリに置かれた\term{draw.io}ファイルは\LaTeX\
処理直前にPDFファイルに変換されてimageディレクトリにコピーされます。

この際、
処理の対象になるのは拡張子が"drawio"であるようなファイルのみです。
Unixファイルシステムは大文字と小文字を区別することに注意してください。
拡張子"draw.io"は無視されます。

